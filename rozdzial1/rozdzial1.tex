\chapter{Wprowadzenie}
\label{cha:wprowadzenie}

Ta praca jest o~uczeniu motywowanym przy eksploracji środowiska przez agenta, 
który w~przypadku tej pracy dyplomowej będzie wykonywana na robocie mobilnym. 
Zastosowane zostanie środowisko symulacyjne umożliwiające sterowanie takim 
robotem, odczytywanie różnych parametrów środowiska otaczającego agenta, 
także tych, których sensory robota, np. kamera, czujniki odległości, nie 
umożliwiają do odczytania.

%---------------------------------------------------------------------------

\section{Cele pracy}
\label{sec:cele_pracy}

Celem poniższej pracy jest zastosowanie uczenia motywowanego do eksploracji 
nieznanego środowiska przez agenta (robota mobilnego) i~porównanie osiągnięć 
takiego agenta w~porównaniu z~podobnym agentem wykorzystującym algorytmy uczenia 
ze wzmocnieniem, np. przy jednoczesnej lokalizacji i~mapowaniu SLAM (ang. 
\textit{simultaneous localization and mapping}).

%---------------------------------------------------------------------------

\section{Zawartość pracy}
\label{sec:zawartosc_pracy}

Ta część będzie napisana jak cała reszta pracy zostanie ukończona\dots
